Sofía Dalia Magallón Páramo was born in Morelia, Michoacán, México. She is currently completing her Bachelor of Science in Computer Science at the Instituto Tecnológico y de Estudios Superiores de Monterrey, where she is set to graduate in December 2023. She has maintained an impressive GPA of 98/100 and was honored with the Líderes del Mañana Distinction, a recognition for her leadership skills and impactful projects.

During her undergraduate studies, Sofía gained valuable professional experience through multiple internships. In the summer of 2023, she interned at Microsoft in Redmond, Washington, as a Software Engineer, where she developed an automation tool that significantly improved the product update process. She also completed internships at Lyft in Mexico City and Oracle in Zapopan, Jalisco, focusing on software engineering and technical troubleshooting, respectively. In the summer of 2021, Sofía was part of a Production Engineer Internship at Meta and Major League Hacking, where she participated in a 12-week educational program and completed projects using a range of technologies including Flask, PostgreSQL, MongoDB, and React.

Sofía has been actively involved in significant projects, such as NavigAbility, an award-winning accessibility navigation app, and Pedagog, an educational platform created during a hackathon. Her technical expertise spans a variety of programming languages and technologies, including Java, JavaScript, Python, HTML, CSS, Git, Linux, ReactJS, NextJS, Flask, NodeJS, MySQL, MongoDB, PostgreSQL, Docker, and Kubernetes.

In addition to her technical achievements, Sofía has demonstrated outstanding leadership qualities. She has been a GitHub Campus Expert since October 2021, organizing computer science events and mentoring at tech gatherings. As President of TECoding since July 2020, she has led initiatives to engage and grow the tech community at her university. Her role as a Peer Mentor and Community Leader from August 2020 to July 2022 further exemplifies her commitment to supporting and guiding fellow students in computer science.