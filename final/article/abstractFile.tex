This study investigates the comparison between different Natural Language Processing (NLP) techniques in detecting emotions from textual data. Given the current state of NLP, and the growing popularity of the Artificial Intelligence (AI) research field, this research aims to shine some light on the field of Sentiment Analysis by comparing the performance of three different techniques: Rules, Neural Network, and Deep Learning, in the correct classification of six different emotions: Sadness, Anger, Love, Surprise, Fear, and Joy, using a dataset of over 16,000 labeled texts. The results show that while Rules are faster to compute and require less resources, models created using Neural Networks and Deep Learning demonstrated a performance improvement of 47\% and 49\% respectively in terms of accuracy against Rules, with a 59\% accuracy. The study concludes that both Neural Networks and Deep Learning approaches are more powerful than Rules, making them more suitable for most applications of Emotion Classification, however, they require more computational resources.